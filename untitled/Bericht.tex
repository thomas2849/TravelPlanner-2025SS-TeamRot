\documentclass[a4paper,12pt]{article}
\usepackage[utf8]{inputenc}
\usepackage{amsmath}
\usepackage{graphicx}
\usepackage{hyperref}

\title{Fachkonzept: TravelPlanner}
\author{}
\date{}

\begin{document}

    \maketitle

    \section*{I. Einleitung}
    \textbf{TravelPlanner} ist eine Webanwendung, die es registrierten Nutzern ermöglicht, Sehenswürdigkeiten in ausgewählten Großstädten zu entdecken und ihre Reise zu planen.
    Die Plattform bietet eine einfache Benutzeroberfläche mit Funktionen wie Benutzerregistrierung, Login, Städtenavigation und einer Suchfunktion.

    \subsection*{A. Motivation}
    Reisen ist für viele ein zentrales Hobby. Oft fehlt es aber an kompakten Tools zur strukturierten Planung von Städtereisen.
    TravelPlanner bietet eine schlanke Lösung für Nutzer, die gezielt attraktive Orte in den beliebtesten Städten finden möchten.

    \subsection*{B. Alleinstellungsmerkmal}
    TravelPlanner konzentriert sich bewusst auf 5 große, internationale Städte, statt eine überladene Plattform zu sein.
    Hauptmerkmale:
    \begin{itemize}
        \item Minimalistisches, schnelles Interface
        \item Fokus auf \textbf{Inspiration und Planung}, nicht auf Buchung
        \item Klare, einfache Nutzerführung
    \end{itemize}

    \section*{II. Verwandte Arbeiten}
    Im Vergleich zu Plattformen wie *TripAdvisor* oder *Google Trips*, die komplex und oft werbelastig sind, bietet TravelPlanner eine fokussierte Alternative für gezielte Reisevorbereitung.

    \section*{III. Anforderungen}

    \subsection*{A. MUSS-Anforderungen (Minimum Viable Product – MVP)}
    \begin{enumerate}
        \item \textbf{Registrieren} \\
        Als neuer Nutzer möchte ich mich registrieren, um später persönliche Funktionen nutzen zu können. \\
        \textit{Akzeptanzkriterien}: Formular mit Username und Passwort, Username muss eindeutig sein, Passwort wird sicher gespeichert (Hashing).

        \item \textbf{Anmelden} \\
        Als registrierter Nutzer möchte ich mich einloggen können. \\
        \textit{Akzeptanzkriterien}: Formular für Username und Passwort, Prüfung und Rückmeldung bei falschen Daten.

        \item \textbf{Städte durchsuchen} \\
        Als Nutzer möchte ich eine Liste der 5 Städte sehen können.

        \item \textbf{Sehenswürdigkeiten anzeigen} \\
        Als Nutzer möchte ich in jeder Stadt die wichtigsten Attraktionen angezeigt bekommen.
    \end{enumerate}

    \subsection*{B. SOLL-Anforderungen}
    \begin{enumerate}
        \item \textbf{Attraktionen speichern} \\
        Als eingeloggter Nutzer möchte ich Attraktionen zu meinen Favoriten hinzufügen können. \\
        \textit{Akzeptanzkriterien}: Button zum Speichern von Attraktionen, Gespeicherte Attraktionen werden mit Nutzer verknüpft.

        \item \textbf{Gespeicherte Orte anzeigen} \\
        Als Nutzer möchte ich eine Übersicht meiner gespeicherten Attraktionen abrufen können.

        \item \textbf{Stadt-Suchfunktion} \\
        Als Nutzer möchte ich über eine Suchleiste nach einer Stadt suchen können.

        \item \textbf{Abmelden} \\
        Als Nutzer möchte ich mich sicher abmelden können.
    \end{enumerate}

    \subsection*{C. OPTIONAL-Anforderungen}
    \begin{enumerate}
        \item \textbf{Attraktion aus Favoriten entfernen} \\
        Als Nutzer möchte ich gespeicherte Attraktionen wieder löschen können.

        \item \textbf{Profilseite} \\
        Als Nutzer möchte ich meine Aktivitäten in einer Profilübersicht sehen können.

        \item \textbf{Responsives Design} \\
        Die Web-App soll auf Mobilgeräten gut nutzbar sein.
    \end{enumerate}

    \section*{IV. Technisches Grobkonzept}
    \subsection*{A. Datenbank – MongoDB}
    Speicherung von:
    \begin{itemize}
        \item Benutzerdaten (Username, Passwort-Hash)
        \item (Optional) Gespeicherten Attraktionen der Nutzer
    \end{itemize}

    \subsection*{B. Backend – Node.js + Express}
    Bereitstellung einer \textbf{REST API} für:
    \begin{itemize}
        \item Registrierung
        \item Login
        \item Abrufen von Städten und Attraktionen
        \item (Optional) Speichern und Abrufen von Favoriten
    \end{itemize}

    \subsection*{C. Frontend – HTML, CSS, JavaScript}
    Grundlegendes Web-Interface mit späterer Option zur Nutzung von \textbf{React.js}.

    \subsection*{D. Hosting – AWS}
    \begin{itemize}
        \item \textbf{EC2 Instanz} für Serverbetrieb
        \item (Optional) \textbf{AWS S3} für statisches Hosting der Webseite
    \end{itemize}

    \section*{V. Akzeptanzkriterien Übersicht}
    \begin{tabular}{|c|c|}
        \hline
        \textbf{Funktion} & \textbf{Akzeptanzkriterium} \\
        \hline
        Registrierung & Einzigartiger Username, Passwort sicher gespeichert \\
        Login & Erfolgreiche Anmeldung oder Fehlermeldung \\
        Stadtübersicht & 5 Städte werden angezeigt \\
        Attraktionen anzeigen & Pro Stadt 5–10 Sehenswürdigkeiten sichtbar \\
        (SOLL) Attraktion speichern & Button vorhanden, Favoritenliste speichert Auswahl \\
        (SOLL) Gespeicherte Orte anzeigen & Persönliche Favoriten werden angezeigt \\
        (SOLL) Stadt suchen & Eingabe filtert Städte dynamisch \\
        \hline
    \end{tabular}

    \section*{VI. Zielsetzung}
    \begin{itemize}
        \item \textbf{Benutzerfreundlichkeit} durch einfache Navigation
        \item \textbf{Sicherheit} durch sichere Passwortspeicherung
        \item \textbf{Skalierbarkeit} für spätere Funktionen
        \item \textbf{Mobile Optimierung} als optionale Erweiterung
    \end{itemize}

\end{document}
